\documentclass[10pt, letterpaper]{article}

% Packages:
\usepackage[
    ignoreheadfoot, % set margins without considering header and footer
    top=2 cm, % seperation between body and page edge from the top
    bottom=2 cm, % seperation between body and page edge from the bottom
    left=2 cm, % seperation between body and page edge from the left
    right=2 cm, % seperation between body and page edge from the right
    footskip=1.0 cm, % seperation between body and footer
    % showframe % for debugging 
]{geometry} % for adjusting page geometry
\usepackage{titlesec} % for customizing section titles
\usepackage{tabularx} % for making tables with fixed width columns
\usepackage{array} % tabularx requires this
\usepackage[dvipsnames]{xcolor} % for coloring text
\definecolor{primaryColor}{RGB}{0, 79, 144} % define primary color
\usepackage{enumitem} % for customizing lists
\usepackage{fontawesome5} % for using icons
\usepackage{amsmath} % for math
\usepackage[
    pdftitle={Alexander Sauceda's CV},
    pdfauthor={Alexander Sauceda},
    pdfcreator={LaTeX with RenderCV},
    colorlinks=true,
    urlcolor=primaryColor
]{hyperref} % for links, metadata and bookmarks
\usepackage[pscoord]{eso-pic} % for floating text on the page
\usepackage{calc} % for calculating lengths
\usepackage{bookmark} % for bookmarks
\usepackage{lastpage} % for getting the total number of pages
\usepackage{changepage} % for one column entries (adjustwidth environment)
\usepackage{paracol} % for two and three column entries
\usepackage{ifthen} % for conditional statements
\usepackage{needspace} % for avoiding page brake right after the section title
\usepackage{iftex} % check if engine is pdflatex, xetex or luatex

% Ensure that generate pdf is machine readable/ATS parsable:
\ifPDFTeX
    \input{glyphtounicode}
    \pdfgentounicode=1
    % \usepackage[T1]{fontenc} % this breaks sb2nov
    \usepackage[utf8]{inputenc}
    \usepackage{lmodern}
\fi



% Some settings:
\AtBeginEnvironment{adjustwidth}{\partopsep0pt} % remove space before adjustwidth environment
\pagestyle{empty} % no header or footer
\setcounter{secnumdepth}{0} % no section numbering
\setlength{\parindent}{0pt} % no indentation
\setlength{\topskip}{0pt} % no top skip
\setlength{\columnsep}{0cm} % set column seperation
\makeatletter
\let\ps@customFooterStyle\ps@plain % Copy the plain style to customFooterStyle
\patchcmd{\ps@customFooterStyle}{\thepage}{
    \color{gray}\textit{\small Alexander Sauceda - Page \thepage{} of \pageref*{LastPage}}
}{}{} % replace number by desired string
\makeatother
\pagestyle{customFooterStyle}

\titleformat{\section}{\needspace{4\baselineskip}\bfseries\large}{}{0pt}{}[\vspace{1pt}\titlerule]

\titlespacing{\section}{
    % left space:
    -1pt
}{
    % top space:
    0.3 cm
}{
    % bottom space:
    0.2 cm
} % section title spacing

\renewcommand\labelitemi{$\circ$} % custom bullet points
\newenvironment{highlights}{
    \begin{itemize}[
        topsep=0.10 cm,
        parsep=0.10 cm,
        partopsep=0pt,
        itemsep=0pt,
        leftmargin=0.4 cm + 10pt
    ]
}{
    \end{itemize}
} % new environment for highlights

\newenvironment{highlightsforbulletentries}{
    \begin{itemize}[
        topsep=0.10 cm,
        parsep=0.10 cm,
        partopsep=0pt,
        itemsep=0pt,
        leftmargin=10pt
    ]
}{
    \end{itemize}
} % new environment for highlights for bullet entries


\newenvironment{onecolentry}{
    \begin{adjustwidth}{
        0.2 cm + 0.00001 cm
    }{
        0.2 cm + 0.00001 cm
    }
}{
    \end{adjustwidth}
} % new environment for one column entries

\newenvironment{twocolentry}[2][]{
    \onecolentry
    \def\secondColumn{#2}
    \setcolumnwidth{\fill, 4.5 cm}
    \begin{paracol}{2}
}{
    \switchcolumn \raggedleft \secondColumn
    \end{paracol}
    \endonecolentry
} % new environment for two column entries

\newenvironment{header}{
    \setlength{\topsep}{0pt}\par\kern\topsep\centering\linespread{1.5}
}{
    \par\kern\topsep
} % new environment for the header

\newcommand{\placelastupdatedtext}{% \placetextbox{<horizontal pos>}{<vertical pos>}{<stuff>}
  \AddToShipoutPictureFG*{% Add <stuff> to current page foreground
    \put(
        \LenToUnit{\paperwidth-2 cm-0.2 cm+0.05cm},
        \LenToUnit{\paperheight-1.0 cm}
    ){\vtop{{\null}\makebox[0pt][c]{
        \small\color{gray}\textit{Last updated in August 2024}\hspace{\widthof{Last updated in August 2024}}
    }}}%
  }%
}%

% save the original href command in a new command:
\let\hrefWithoutArrow\href

% new command for external links:
\renewcommand{\href}[2]{\hrefWithoutArrow{#1}{\ifthenelse{\equal{#2}{}}{ }{#2 }\raisebox{.15ex}{\footnotesize \faExternalLink*}}}


\begin{document}
    \newcommand{\AND}{\unskip
        \cleaders\copy\ANDbox\hskip\wd\ANDbox
        \ignorespaces
    }
    \newsavebox\ANDbox
    \sbox\ANDbox{}

    \placelastupdatedtext
    \begin{header}
        \textbf{\fontsize{24 pt}{24 pt}\selectfont Alexander Sauceda}

        \vspace{0.3 cm}

        \normalsize
        \mbox{{\color{black}\footnotesize\faMapMarker*}\hspace*{0.13cm}Aurora, CO}%
        \kern 0.25 cm%
        \AND%
        \kern 0.25 cm%
        \mbox{\hrefWithoutArrow{mailto:alexander@alexandersauceda.dev}{\color{black}{\footnotesize\faEnvelope[regular]}\hspace*{0.13cm}alexander@alexandersauceda.dev}}%
        \kern 0.25 cm%
        \AND%
        \kern 0.25 cm%
        \mbox{\hrefWithoutArrow{tel:+1-714-504-6677}{\color{black}{\footnotesize\faPhone*}\hspace*{0.13cm}(714) 504-6677}}%
        \kern 0.25 cm%
        \AND%
        \kern 0.25 cm%
        \mbox{\hrefWithoutArrow{https://alexandersauceda.dev/}{\color{black}{\footnotesize\faLink}\hspace*{0.13cm}alexandersauceda.dev}}%
        \kern 0.25 cm%
        \AND%
        \kern 0.25 cm%
        \mbox{\hrefWithoutArrow{https://linkedin.com/in/alexander-matthew-sauceda}{\color{black}{\footnotesize\faLinkedinIn}\hspace*{0.13cm}alexander-matthew-sauceda}}%
        \kern 0.25 cm%
        \AND%
        \kern 0.25 cm%
        \mbox{\hrefWithoutArrow{https://github.com/alesauce}{\color{black}{\footnotesize\faGithub}\hspace*{0.13cm}alesauce}}%
    \end{header}

    \vspace{0.3 cm - 0.3 cm}


    \section{Summary}



        
        \begin{onecolentry}
            A dedicated and versatile backend developer looking to utilize API-driven development, system architecture, and software engineering skills learned through 8+ years with Amazon.
        \end{onecolentry}


    
    \section{Experience}



        
        \begin{twocolentry}{
        \textit{Seattle, WA}    
            
        \textit{Aug 2021 – present}}
            \textbf{Systems \& Software Engineer}
            
            \textit{Amazon Photos, Trust \& Safety Engineering}
        \end{twocolentry}

        \vspace{0.10 cm}
        \begin{onecolentry}
            \begin{highlights}
                \item Overhauled legacy Python codebase to comply with new regulations, reducing process defects by \textasciitilde{}89\% YoY.
                \item Designed and implemented new service to automate handling of customer account issues, reducing engineer hours spent on manual processes by 25\%. Utilized Python, AWS Lambda, DynamoDB, and SQS as well as Cloud Development Kit (CDK) for infrastructure as code.
                \item Designed and currently implementing a new Java-based service for abuse report automation and API to integrate with customer service systems; projected to reduce engineer time on manual investigations by 10\%.
                \item Earned AWS Certified Solutions Architect - Associate certification.
                \item Oversaw service operations, including CI/CD pipelines, CloudWatch monitors, resolving high-severity incidents and providing root cause analysis as part of team’s on-call rotation.
                \item Developed a new SQL-based process for investigating customer file issues, reducing mean time to resolution by 50\%.
            \end{highlights}
        \end{onecolentry}


        \vspace{0.2 cm}

        \begin{twocolentry}{
        \textit{Seattle, WA}    
            
        \textit{Aug 2020 – Aug 2021}}
            \textbf{Support Engineer}
            
            \textit{Amazon Explore, Business Development}
        \end{twocolentry}

        \vspace{0.10 cm}
        \begin{onecolentry}
            \begin{highlights}
                \item Designed and implemented new analytics tooling, enabling business development teams to proactively identify and correct poor customer experiences, resulting in a customer satisfaction score increase of \textasciitilde{}4\%.
                \item Developed a new onboarding guide for partners selling “experiences” and reduced onboarding time by 10\% for new sellers.
                \item Investigated customer-facing issues and provided root-cause analysis to the software development team which resulted in a reduction of software-related defects from 9\% to 1.5\%.
                \item Developed a new org standard process for remediating seller connectivity issues.
            \end{highlights}
        \end{onecolentry}


        \vspace{0.2 cm}

        \begin{twocolentry}{
        \textit{Seattle, WA/Phoenix, AZ}    
            
        \textit{June 2016 – Aug 2020}}
            \textbf{Program Manager/Area Manager}
            
            \textit{Amazon Fulfillment/Supply Chain Execution}
        \end{twocolentry}

        \vspace{0.10 cm}
        \begin{onecolentry}
            \begin{highlights}
                \item Streamlined reporting process for fulfillment network fullness by leveraging SQL skills, which reduced manual bridging time by \textasciitilde{}5 hours/week.
                \item Automated team reports using SQL, Excel, and AWS Quicksight, saving \textasciitilde{}20 hours/month.
                \item Executed multiple projects to reconfigure racking, adding 70,933 cubic feet of storage and saving \textasciitilde{}\$1,053,940 compared to building new racking.
                \item Oversaw utilization and stow strategy, setting a new building inventory record with a 15.8\% improvement, resulting in 110\% storage utilization and \textasciitilde{}\$900,000 cost savings.
            \end{highlights}
        \end{onecolentry}



    
    \section{Education}



        
        \begin{twocolentry}{
            
            
        \textit{Aug 2012 – May 2016}}
            \textbf{University of Arizona}

            \textit{BS in Information Sciences}
        \end{twocolentry}




    
    \section{Skills}



        
        \begin{onecolentry}
            \textbf{Technologies:} AWS (Lambda, SQS, DynamoDB, CDK), Python, SQL, Java, Nix, Git version control, CI/CD pipelines, Amazon-internal software build system
        \end{onecolentry}


    

\end{document}